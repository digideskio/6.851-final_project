\documentclass[11pt]{article}
\usepackage[utf8]{inputenc}

\usepackage{ifpdf}
\ifpdf
\usepackage[pdftex]{graphicx}
\else
\usepackage{graphicx}
\fi

\usepackage[margin=1.25in]{geometry}

\title{6.851 Final Project \\ Tabulation Hashing Performance Benchmark}
\author{Maksim Stephenako \\ Yuzhi Zheng}

\date{May 2012}

\begin{document}

\setlength{\baselineskip}{1\baselineskip}

\ifpdf
\DeclareGraphicsExtensions{.pdf, .jpg, .tif}
\else
\DeclareGraphicsExtensions{.eps, .jpg}
\fi

\maketitle

\section{Introduction}
% Yuzhi
Hashing is one of the most basic computer science concept. 
It allows elements to be reliably stored and 
retrieved from a limited number of slots, without dedicated slot of every possible 
variation of the element. While basic, hashing is used everywhere. Hashing is used in 
associative arrays, sometimes also known as dictionaries, in languages like 
PHP, Perl, and Python. Hashing can event be used for database indexing. 
Even lower level computer architectural components like processor caches 
use ideas from hashing to figure out which line to store value from a particular
memory address. Hashing can also be used to keep track of sets or make 
sure certain data representations are unique. Even the famous MapReduce
framework uses hashing to help shard inputs to be processed on different machines.

From a theoretical standpoint, hashing takes $O(1)$ time, which means it takes a constant
amount of time. That is essentially as fast as it gets. However, big-O notations can not
accurately depict the size of the constant factor. These constant factors sometimes
have a significant but real influence on the performance of any algorithm. 
Since hashing is used so often, it is important to keep that constant factor
as low as possible, and finding improvements whenever possible.

One of the most basic hashing function is the multiplicative hashing. 
Thorup and Zhang showed that a different type of hashing, tabulation hashing,
could potentially be a good alternative to the more basic multiplicative hashing 
in their paper from 2010. More specifically, they looked at the performance of tabulation hashing
used in conjunction with linear probing and found the performance to be competitive with
other hash functions on dense tables.

This report takes a closer look at tabulation hashing and it's performance 
against the basic multiplicative hashing. Instead of only looking at linear 
probing, we expanded our collision resolution techniques to quadratic probing 
and also chaining. We plan to do some benchmark testing as well as  
analyzing the possible pros and cons of each type of hash functions as 
well as the different collision resolutions.


\section{Tabulation Hashing}
% Maksim
overall idea of tabulation hashing

make table

look stuff up etc
 
- 3 independence
 
- 4 independence

- 5 independence
\section{Implementation}
% Yuzhi

We implemented this project in C, hoping the result will be fast and efficiently.
We enjoyed knowing exactly where certain arrays and variables are going to
be laid out in memory. In the end, we have approximately 1.5k lines of code,
including the hash functions, table generation, collision detection, and test code.

Fortunately for us, Thorup and Zhang included the code for tabulation hashing in their
2010 paper on 5-independent tabulation hashing. We were able to model most of our
code based on what was included in the paper. We kept the logic behind how the hashes
are generated, but made some changes on how the structures are stored in the code. 
Storing fewer pointers, hoping that will use less memory space and have higher performance.

\subsection{Random Numbers}

Tabulation hashing requires tables and tables of random numbers in ordering 
to function correctly. The C language's standard \texttt{rand()} function only
guarantees up to 15 bits of random bits. However, we needed at least 32-bit
or 64-bit for each entry in our random number tables. Thus, we recreated our 
own version of random number generator by calling the \texttt{rand()} function
and number of times and shifting the randomly generated bits. Even though
the \texttt{rand()} function is only a pseudorandom number generator, we thought
it should be good enough for our purpose. We made sure to seed 
the \texttt{rand()} function each time we run our program.

\subsection{Hash Functions}
We had a 5 hash functions. One is a basic multiplicative function and the four other ones are some variation of the tabulation hash function.
\subsubsection{Univ2}
This is the basic multiplicative hashing. It takes a value to hash, multiply it by a number
and then adds another number to generate a 32-bit hash. 

\subsubsection{Short32}
This is a tabulation hashing function. It divides up the 32-bits into 16-bit (\texttt{short}) chunks. 
It has a look up table for each chunk, as well as the sum of the chunks.  
This requires a total of 3 random number tables. 

\begin{table}
\centering 
\begin{tabular}{|c|c|}
  \hline
T0 & $2^{16}\times4$ bytes\\  \hline
T1 & $2^{16}\times4$ bytes\\ \hline
T2 & $2^{17}\times4$ bytes\\
  \hline \hline
  Total & 1 megabyte \\
  \hline
\end{tabular}
\caption{Space utilized by tables for Short32}
\label{tab:short32mem}
\end{table}

\subsubsection{Char32}
This is also a tabulation hashing function. It divides up the 32-bits into 
four 8-bit (\texttt{char}) chunks. There is a look up table for each of the
chunks and a few extra table for additional generated characters. 
This requires a total of 7 random number tables and 7 table look-ups. 
Some look-ups uses more than 1 random number from the table.

\begin{table}
\centering 
\begin{tabular}{|c|c|}
  \hline
T0 & $2^{8}\times 2 \times4$ bytes\\  \hline
T1 & $2^{8}\times 2 \times4$ bytes\\ \hline
T2 & $2^{8}\times 2 \times4$ bytes\\  \hline
T3 & $2^{8}\times 2 \times4$ bytes\\ \hline
T4 & $2^{10}\times4$ bytes\\  \hline
T5 & $2^{10}\times4$ bytes\\ \hline
T6 & $2^{11}\times4$ bytes\\
  \hline \hline
  Total & 32 kilobytes \\
  \hline
\end{tabular}
\caption{Space utilized by tables for Char32}
\label{tab:char32mem}
\end{table}

\subsubsection{Short64}
Short64 is a hash function that divides a 64-bit key into 4 chunks of 16-bits.
The actual algorithm is similar to Char32, except this function has much 
larger tables, even though it has the same number of tables.

\begin{table}
\centering 
\begin{tabular}{|c|c|}
  \hline
T0 & $2^{16}\times 2 \times8$ bytes\\  \hline
T1 & $2^{16}\times 2 \times8$ bytes\\ \hline
T2 & $2^{16}\times 2 \times8$ bytes\\  \hline
T3 & $2^{16}\times 2 \times8$ bytes\\ \hline
T4 & $2^{21}\times8$ bytes\\  \hline
T5 & $2^{21}\times8$ bytes\\ \hline
T6 & $2^{22}\times8$ bytes\\
  \hline \hline
  Total & 68 megabytes \\
  \hline
\end{tabular}
\caption{Space utilized by tables for Short64}
\label{tab:short64mem}
\end{table}

\subsubsection{Char64}
This is the most complicated tabulation hash function we have.
It requires 15 lookup tables and also the most number of table
accesses. However, since each chunk is only 8-bits the total
size of the lookup tables is actually much smaller than that of short64.

\begin{table}
\centering 
\begin{tabular}{|c|c|}
  \hline
T0 & $2^{8}\times(1+1+0.5) \times8$ bytes\\  \hline
T1 & $2^{8}\times(1+1+0.5) \times8$ bytes\\ \hline
T2 & $2^{8}\times (1+1+0.5) \times8$ bytes\\  \hline
T3 & $2^{8}\times (1+1+0.5) \times8$ bytes\\ \hline
T4 & $2^{8}\times(1+1+0.5) \times8$ bytes\\  \hline
T5 & $2^{8}\times(1+1+0.5) \times8$ bytes\\ \hline
T6 & $2^{8}\times (1+1+0.5) \times8$ bytes\\  \hline
T7 & $2^{8}\times (1+1+0.5) \times8$ bytes\\ \hline
T8 & $2^{11}\times8$ bytes\\  \hline
T9 & $2^{11}\times8$ bytes\\ \hline
T10 & $2^{11}\times8$ bytes\\  \hline
T11 & $2^{11}\times8$ bytes\\ \hline
T12 & $2^{21}\times8$ bytes\\  \hline
T13 & $2^{11}\times8$ bytes\\ \hline
T14 & $2^{21}\times8$ bytes\\
  \hline \hline
  Total & $\approx$ 32 megabytes \\
  \hline
\end{tabular}
\caption{Space utilized by tables for Char64}
\label{tab:char64mem}
\end{table}

\subsection{Collision Resolution}
problems we ran into

counting collisions instead of absolute time ( or maybe we can do both)
\subsection{Small Improvements}

\section{Benchmark Results}
% Yuzhi

Compare pure hashing vs tabulation hashing

compare linear probing

compare quadratic probing

compare chaining

compare between the three

some analysis on memory access  and mention pros ad cons of each


\section{Conclusion}
% Maksim
summrize what we wanted to find out

what we did

and the results we found

\bibliographystyle{plain}
% Stuff will show up here if you use BibTeX
\begin{thebibliography}{77}

\bibitem{tz2010}
M. Thorup, Y. Zhang
\emph{Tabulation Based 5-Universal Hashing and Linear Probing},
2010

\bibitem{tz2004}
M. Thorup and Y. Zhang 
\emph{Tabulation Based 4-Universal Hashing with Applications to Second Moment Estimation},
Proc. 15th SODA:608-617 2004.


\end{thebibliography}

\bibliography{}

\end{document}

